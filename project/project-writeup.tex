\documentclass[11pt]{article}
\usepackage{fullpage}
\usepackage{enumitem}
\usepackage{pdfpages}
\usepackage{graphicx}
\usepackage{amsmath, amsthm}
\usepackage{algorithm}
\usepackage{algorithmicx}
\usepackage{algpseudocode}
\usepackage{booktabs}
%\setlength{\parindent}{4em}
%\setlength{\parskip}{1em}
%\renewcommand{\baselinestretch}{1.5}

\begin{document}
\author{Benjamin Sorenson} \title{Final Project}
\maketitle

\begin{abstract}
  As a matter of test security (read, cheating deterrent). The state
  of Pennsylvania has required the use of scrambled test forms for its
  high-stakes K-12 assessments, the PSSA and Keystone Exams. Given the
  requirements that the each scrambled form still meet the test
  blueprint set out by the state,
\end{abstract}
\section{Description}
\par
As a matter of test security, Pennsylvania requires that both its
summative assessment PSSA and its graduation requirement Keystone
exams are scrambled. This is to prevent both student to student
cheating as deter administrator manipulation of answer documents
(e.g., correcting answers after the fact).
\par
When each test is constructed, test development specialists must
guidlines ensuring psychometric and content requirements are met. For
example, one guideline may be that for multiple choice questions there
can be no sequences of 4 or more of the same answer key in a row.
Another might be that the first question on the test must be of medium
to low difficulty.  When constructing scrambled forms, care must be
taken so that these test construcion guidelines are followed to the
extent that it is possible.
\par
Prior to programmatic scrambling,the work of constructing scrambled
forms was done by hand, often incurring a significant cost to the
state of Pennsylvania as members of the state education board would
have to travel to Minnesota to work with content specialists, and
determine the best scrambling patterns meeting the test
blueprint. This process could take weeks to complete, and for this
reason, this problem is of practical interest.
\par
Form scrambling happens in two stages.  First, content and test
development specialists construct a single, ideal form called the
``Master Core''.  Then, using the following guidelines, they split the
Master Core into blocks of multiple choice items. Once these blocks
have been constructed, they are passed off, and assembled in to 7
additional unique forms.

The complete description from 2014 PSSA technical report
\begin{itemize}
\item Items cannot move between blocks.
\item DRC and PDE content specialists will work to ensure that the
  scrambling does not result in making content more difficult than the
  Master Core item sequence. For example, items of similar cognitive
  complexity will be swapped rather than random scrambling.
\item A block scramble pattern is only valid if it does not contain an
  invalid key distribution within the block. Additional checks for an
  invalid key distribution across blocks must be made when combining
  block scramble patterns to create forms. For example, scrambling
  must not create more than three (3) of the same key positions in a
  row.
\item A block scramble pattern is only valid if it does not contain an
  invalid standard (AA/EC) distribution within a block. Additional
  checks for standard distribution across blocks must be made when
  combining block scramble patterns to create forms.  An exception was
  made for one mathematics scramble for each grade which ordered items
  within block by eligible content per PDE request.  Scrambling should
  not place a difficult item as the first item in a section. The first
  item in a block that does NOT begin a section may be a difficult
  item since blocks are invisible to the student.
\item For passage-based items, a block scramble pattern is only valid
  if it does not create dissonance between the items and passage(s).
\item Scrambling should not place a difficult item as the first item
  in a passage set.
\item Within a set of items connected to a paired set of passages, an
  item associated with both passages can be swapped only with another
  item associated with both passages.  (These items must remain at the
  end of the set of items associated with the passage set.)

\end{itemize}

\par
In addition to the constraints above, further constraints are placed
on the block assembly to ensure that an application of incorrect key
can be easily detected. For this reason, the pair-wise answer key
similarity between forms is restricted. Ideally, each form's key would
overlap with any other form's key in the pool of forms no more than
chance.  In practice, however, because the scrambling of each block is
not random, it is difficult to meet this ideal given a sufficiently
large pool. This is partly due to the fact that the number of pairs in
a given pool increases quadratically, and partly due to the fact that
scrambling within blocks is not random, and key distributions are not
random within blocks.  Given this, the usual guideline is that answer
key overlap between any two forms not exceed the lowest performance
cut.  Although, the state of Pennsylvania does give out letter grades,
this requirement is equivalent to saying that the misapplication of an
answer key to a form must never result in a grade higher than an
``F''.
\par
The current approach to solving this problem is by randomized
depth-first search using a matrix representation.  Each row of the
matrix represents a scrambled form, and each column corresponds to a
block number.  The value in each cell represents a scramble pattern.
For example the matrix
\[
A =
\begin{bmatrix}
  0 & 0 & 0 \\
  1 & 0 & 2 \\
  2 & 1 & 1
\end{bmatrix}
\]
would represent a solution to problem in which there were 2 forms
needed in addition to the Master Core, and \(a_{21}=1\)
means that the form-1 used scramble pattern 1 for block 1. Zeros
indicate the use of a Master Core (unscrambled) block.  For instance
\(a_{22} = 0\)
indicates that block-2 of form-1 is an unscrambled block.
\begin{itemize}
\item Each block scramble pattern must be used once.
\item No block scramble pattern can be used more than a user-specified
  number of times
\end{itemize}
\begin{algorithm}
  \begin{algorithmic}
   
    \Function{\textsc{Scramble}}{$\mathit{max\_restarts}$}\State
    $\mathit{more\_steps\_back} \gets 0$ \State
    $\mathit{restarts} \gets 0$ \State
    $\mathit{assignment} \gets [(0, 0, \dots ,0)]$ \State
    $\mathit{forced\_choice} \gets 0$ \Comment
    $\mathit{forced\_choice}$ is set by
    \textsc{GetEligibleBlockScrambles}
    \While{$\mathit{restarts} < \mathit{max\_restarts}$} \For{$i = 1$
      \textbf{to} $\mathit{max\_iterations}$} \State
    $\mathit{eligible\_block\_scrambles} \gets$
    \Call{\textsc{GetEligibleBlockScrambles}}{$\mathit{assignment}$}
    \If{$\text{there are no choices for one or more blocks}$}
    \textbf{break}\Else \State
    $\mathit{scramble\_pattern} \gets \text{random choice from each
      eligible block}$
    \EndIf
    \If{$\mathit{scramble\_pattern}$ satisfies constraints} \State add
    $\mathit{scramble\_pattern}$ to $\mathit{assignment}$
    \EndIf \If{all forms assinged} \State \Return
    $\mathit{assignment}$ \EndIf
    \EndFor
    \If{$\mathit{forced\_choice} > 0$}
    \If{$\mathit{forced\_choice}-\mathit{more\_steps\_back} > 1$}
    \State remove all but the first
    $\mathit{forced\_choice} - \mathit{more\_steps\_back}$ forms from
    $\mathit{assignment}$ \State
    $\mathit{more\_steps\_back} = \mathit{more\_steps\_back} + 1$
    \Else\State $\mathit{assignment} \gets [(0, 0, \dots ,0)]$ \State
    $\mathit{restarts} = \mathit{restarts} + 1$ \State
    $\mathit{more\_steps\_back} = 0$
    \EndIf\State
    $\mathit{forced\_choice} = 0$
    \ElsIf{$\mathit{assingment.length} - \mathit{more\_steps\_back} >
      1$}
    \State remove all but the first
    $\mathit{assignment.length} - \mathit{more\_steps\_back} - 1$
    forms from $\mathit{assignment}$ \State
    $\mathit{more\_steps\_back} = \mathit{more\_steps\_back} + 1$
    \Else \State $\mathit{assignment} \gets [(0, 0, \dots ,0)]$ \State
    $\mathit{restarts} = \mathit{restarts} + 1$ \State
    $\mathit{more\_steps\_back} = 0$
    \EndIf
    \EndWhile \State \Return \textbf{None}
    \EndFunction
  \end{algorithmic}
\end{algorithm}

The idea behind \textsc{Scramble} is that at the shallower depths, it
is easy to find many assignments that will likely satisfy the
constraints, but it is also easy to unwittingly paint yourself into a
corner along the way.  It's difficult to know for sure which variable
assignment caused you to fail, but it's not worth the effort to keep
checking each possible variable at the current depth and backtrack one
level.  In this representation of the problem, for a number of desired
forms (variables) \(n_f\),
number of blocks \(b\),
and number of scramble patterns per block \(s\)
(assuming a equal number of scramble patterns per block), the
branching factor is \(s^b\),
and the depth of the solution is \(n_f\).
So, rather, than search through each \(s^b\)
solution, \textsc{Scramble} lets the user set the number of iterations
before the algorithm gives up and decides that it's \emph{probably} at
a dead-end.  Once it has determined that it's likely at a dead-end, it
first backs up one level, but it also keeps track of how many times
it's had to back up---each time, it backs up one more level than the
previous time.  Once it has backed up all the way to the root, this is
counted as a restart. After a user-specified maximum number of
restarts, the algorithm can either return \textbf{None} or the current
partial assignment---for this project, it was set to return
\textbf{None}.
\par
While it is not possible to know for sure what assignment likely
caused it to run into a dead-end, the function
\textsc{GetEligibleBlockScrambles} employs a simple heuristic to help
determine how far we should backtrack.  When it is called on the
current assignment, it counts the number of block scrambles that are
still available to use, because of the constraint that all scrambles
must be used at least once, it applies to pigeon hole principle to
determine if there are any blocks that \emph{must} be used.  If there
are any, it only returns those as the eligible or legal choices for
the next assignment(s), and indicates where this first occurred. If no
solution is found, then \textsc{Scramble} backs up to the point marked
by \textsc{GetEligibleScrambles} plus any additional levels indicated
by $\mathit{more\_steps\_back}$.
\par
The purpose of this project was to improve on the current algorithm by
employing backtracking search.  I chose the to use the software
aima-python provided by Peter Norvig.  I changed the representation of
the problem so that each (form, block) pair is a single variable. This
new representation changes the problem slightly. I had a difficult
time figuring what should and shouldn't be considered a neighbor \dots
in some sense every variable is a neighbor with every other variable
since there are global constraints that care about the assignment as a
whole---or rather the pool of forms.  Eventually (after some
experimentation) I landed on the variable representation of (form,
block) pairs where instead of representing indices of a matrix, the
pair represents an atomic variable stored in a dictionary.  Neighbors
are those (form, block) pairs whose answer keys share a border with
\begin{enumerate}
\item While the number of restarts is less than the maximum number of
  allowed restarts \dots
  \begin{enumerate}
  \item randomly choose a complete form by randomly selecting a block
    scramble from each of the eligible block scrambles. If there are
    more or as many additional forms needed than there are eligible
    block scrambles, only unused block scrambles are considered
    eligible.  Otherwise, all block scrambles that have been used
    fewer than the maximum number of times allowed are considered
    eligible.
  \item If the selected form satisfies the border and key overlap
    constraints, add it to the solution, and select another
  \item If a complete solution is found, return it
  \item Otherwise \dots
    \begin{enumerate}
    \item If there was a forced choice during the construction of the
      current partial solution, and jumping back to the partial
      solution where the first forced choice occurred plus any
      additional steps back---if this means you jump all the way back
      to the root, count it as a restart, otherwise increment the
      \textsc{MoreStepsBack} counter.
    \item If the start
    \end{enumerate}
  \end{enumerate}
\item
\end{enumerate}

\section{Background}
\par
In response to test security issues raised in prior PSSA
administrations, multiple scrambled patterns of operational forms were
constructed for each mathematics, reading, and science assessment
since 2013. The core form was constructed following the past test
construction and equating guidelines and will be referred to as the
Master Core throughout the remainder of this document. Based on
previous TAC recommendation, the Master Core is the pattern of the
test that would have been administered to all students in the absence
of scrambling. More importantly, the data obtained from administration
of the Master Core were used for operational MC item calibration for
the 2014 PSSA.
\par
Once the Master Core was constructed and approved, DRC and PDE content
specialists built seven scrambled patterns of the Master Core for each
content and grade. OE items were not scrambled so each OE item
appeared in the same position on every form. Some MC items also appear
in the same position on multiple forms due to content constraints. In
some content areas and grades the number of field-test forms was
greater than the number of scrambled patterns. In these instances the
Master Core and scrambled patterns were repeated with no specific
pattern appearing more than three times. Due to the limited enrollment
for the PSSA CBT, only three forms were offered. These forms included
the accommodation form, a Master Core form, and one additional
scrambled form; therefore, these forms have slightly higher
participation than other forms.
\par
When the Master Core was built, the linking position rules were
observed for all core-linking and equating block items. The Master
Core was used at least as often, or more often, than any scrambled
version of the core form. Since form 1 was used for all accommodated
forms (e.g., Braille, Large Print, Audio, and Spanish) it was never
designated as a Master Core. The specific forms presenting the Master
Core vary across grades within each content area. For example, the
Master Core for mathematics Grade 8 is repeated on Forms 2, 9, and 13
while the Master Core for Grade 7 is repeated on Forms 2, 10, and
11. Given that all forms were spiraled at the student level, the
distribution of forms is reasonably uniform. The exception is Form 1,
which had higher participation due to the fact that it is the only
form used for accommodations.
\par
Based on TAC recommendations to minimize possible item position
effects, each section of the Master Core was divided into blocks of
non-overlapping MC items. Recall that the OE items were not part of
the scrambling. The blocks generally contained six to seven MC items
(or one passage), but the block sizes varied depending on the content
and test session. Within each block, MC items were scrambled following
general psychometric and content guidelines to create up to five
versions of the block in addition to the Master Core sequencing. The
blocks were assembled to create seven scrambled versions of the Master
Core. Table 9–3 shows the mathematics Grade 8 scrambled form
structure. The core was divided into nine blocks (labeled “1”–“9”) and
each block was scrambled in five different permutations (labeled I,
II, III, IV, and M). So, for example, Form 1 was constructed with
scrambled block version “I” for all nine blocks. Seven scrambled
variations (labeled A, B, C, D, E, F, and G in the “Pattern” column)
of the Master Core were used in addition to the Master Core across the
twenty field-test forms. The Master Core was used on Forms 2, 9, and
13.
\par
The problem of test scrambling arose out of the need to prevent
answer-copying between students, and make it more difficult for
administrators to change student answers on a large scale after the
test had been completed. When constructing standardized tests, often
there is an overall test blueprint that test developers must adhere to
when selecting items. In addition to content constraints, there are
constraints such as the overall difficulty of the form or the key
distribution (e.g., too many of the same answer key in a row, may
cause some students to change otherwise correct answers).
\par
When constructing a scrambling plan several factors were considered.
\begin{enumerate}
\item How easily can we detect a misapplied key? That is, the answer
  keys between any pair of test forms must be different enough so that
  the misapplication of one form's answer key to another cannot result
  in a passing score.  The pair-wise dissimilarity between forms is
  also desirable because it means that an unskilled cheater cannot
  copy from someone using a different form and get a passing score.
\item The difficulty of each item must remain as unaffected as
  possible by the change in position.  Students tend to get fatigued
  through the administration of an exam so moving an item
\end{enumerate}
\section{Problem Description}
\par
Prior to scrambling the Master Core, DRC and PDE content specialists
developed the following general psychometric and content guidelines:
\begin{itemize}
\item Items cannot move between blocks.
\item DRC and PDE content specialists will work to ensure that the
  scrambling does not result in making content more difficult than the
  Master Core item sequence. For example, items of similar cognitive
  complexity will be swapped rather than random scrambling.
\item A block scramble pattern is only valid if it does not contain an
  invalid key distribution within the block. Additional checks for an
  invalid key distribution across blocks must be made when combining
  block scramble patterns to create forms. For example, scrambling
  must not create more than three (3) of the same key positions in a
  row.
\item A block scramble pattern is only valid if it does not contain an
  invalid standard (AA/EC) distribution within a block. Additional
  checks for standard distribution across blocks must be made when
  combining block scramble patterns to create forms.  An exception was
  made for one mathematics scramble for each grade which ordered items
  within block by eligible content per PDE request.  Scrambling should
  not place a difficult item as the first item in a section. The first
  item in a block that does NOT begin a section may be a difficult
  item since blocks are invisible to the student.
\item For passage-based items, a block scramble pattern is only valid
  if it does not create dissonance between the items and passage(s).
\item Scrambling should not place a difficult item as the first item
  in a passage set.
\item Within a set of items connected to a paired set of passages, an
  item associated with both passages can be swapped only with another
  item associated with both passages.  (These items must remain at the
  end of the set of items associated with the passage set.)

\end{itemize}
\section{Experimental Design}
To test the two algorithms, I chose a set of some of the more
difficult real-world data, and ran each algorithm under varying
conditions.  The conditions covered are:
\begin{itemize}
\item The total number of forms was varied between 5 and 20 stepping
  by 3
\item The total amount of answer-key overlap varied between 25\% and
  50\% of the total answer-key string stepping by 5\%
\item For each of the above conditions, AIMA version of the algorithm
  was run using a least-used-variable heuristic and variable
  shuffling---once with neither, once with variable shuffling only,
  once with least-used-variable heuristic only, and once with both
\item Since the AIMA conditions don't have any effect on the
  performance of the production version, and the production version
  relies on random choices at each node, each of the conditions for
  the AIMA version served as an opportunity for a replication of the
  production version's results
\end{itemize}
The each combination of conditions mentioned above were 4 times
(because that's how many processors I have \dots), for a total of
\(4 \time 2 \times 2 \times 5 \times 6 = 2400 \)
total runs constituting 4 replications for each combination of
conditions for the AIMA version, and 16 replications of each condition
for production version. The results are summarized in tables Most
conditions have no effect on the current implementation, so, they were
used as a replications for the current algorithm.  Based on experience
its performance can vary greatly, on the same data set so looking at a
single run under a single condition, can paint an unrealistic picture
of it's performance.
\par
\begin{table}
  \centering
  \begin{tabular}{llrrrr}
    \toprule
    &      &   time &      &        &        \\
    &      &   mean &  std &    min &    max \\
    \midrule
    5  & 0.25 &        &      &        &        \\
    & 0.30 & 240.00 & 0.00 & 240.00 & 240.00 \\
    & 0.35 &   0.13 & 0.01 &   0.12 &   0.14 \\
    & 0.40 &   0.14 & 0.05 &   0.10 &   0.20 \\
    & 0.45 &   0.10 & 0.02 &   0.08 &   0.11 \\
    & 0.50 &   0.11 & 0.00 &   0.10 &   0.11 \\
    \hline
    8  & 0.25 &   0.11 & 0.01 &   0.09 &   0.11 \\
    & 0.30 & 240.00 & 0.00 & 240.00 & 240.00 \\
    & 0.35 & 240.00 & 0.00 & 240.00 & 240.00 \\
    & 0.40 & 240.00 & 0.00 & 240.00 & 240.00 \\
    & 0.45 &   0.60 & 0.02 &   0.57 &   0.62 \\
    & 0.50 &   0.48 & 0.05 &   0.44 &   0.53 \\
    \hline
    11 & 0.25 &   0.43 & 0.02 &   0.41 &   0.45 \\
    & 0.30 & 240.00 & 0.00 & 240.00 & 240.00 \\
    & 0.35 & 240.00 & 0.00 & 240.00 & 240.00 \\
    & 0.40 & 240.00 & 0.00 & 240.00 & 240.00 \\
    & 0.45 &   3.77 & 0.05 &   3.71 &   3.82 \\
    & 0.50 &   1.55 & 0.11 &   1.39 &   1.64 \\
    \hline
    14 & 0.25 &   1.13 & 0.07 &   1.06 &   1.20 \\
    & 0.30 & 240.00 & 0.00 & 240.00 & 240.00 \\
    & 0.35 & 240.00 & 0.00 & 240.00 & 240.00 \\
    & 0.40 & 240.00 & 0.00 & 240.00 & 240.00 \\
    & 0.45 & 240.00 & 0.00 & 240.00 & 240.00 \\
    & 0.50 &   3.01 & 0.08 &   2.93 &   3.11 \\
    17 & 0.25 &   2.77 & 0.10 &   2.70 &   2.92 \\
    \hline
    & 0.30 & 240.00 & 0.00 & 240.00 & 240.00 \\
    & 0.35 & 240.00 & 0.00 & 240.00 & 240.00 \\
    & 0.40 & 240.00 & 0.00 & 240.00 & 240.00 \\
    & 0.45 & 240.00 & 0.00 & 240.00 & 240.00 \\
    & 0.50 &   6.80 & 0.60 &   6.17 &   7.61 \\
    \hline
    20 & 0.25 &   6.09 & 0.36 &   5.59 &   6.41 \\
    & 0.30 & 240.00 & 0.00 & 240.00 & 240.00 \\
    & 0.35 & 240.00 & 0.00 & 240.00 & 240.00 \\
    & 0.40 & 240.00 & 0.00 & 240.00 & 240.00 \\
    & 0.45 & 240.00 & 0.00 & 240.00 & 240.00 \\
    & 0.50 & 240.00 & 0.00 & 240.00 & 240.00 \\
    \bottomrule
  \end{tabular}
\end{table}
And then there is some text ...
\begin{table}
  \centering
  \begin{tabular}{llrrrrr}
    \toprule
    &      &  time &        &       &        &        \\
    &      & count &   mean &   std &    min &    max \\
    nforms & pct\_key\_overlap &       &        &       &        &        \\
    \midrule
    5  & 0.25 &    16 &  15.94 &  8.91 &   2.18 &  26.96 \\
    & 0.30 &    16 &   0.56 &  0.99 &   0.00 &   3.88 \\
    & 0.35 &    16 &   0.19 &  0.45 &   0.00 &   1.61 \\
    & 0.40 &    16 &   0.06 &  0.24 &   0.00 &   0.97 \\
    & 0.45 &    16 &   0.20 &  0.28 &   0.00 &   0.83 \\
    & 0.50 &    16 &   0.22 &  0.39 &   0.00 &   1.23 \\
    \hline
    8  & 0.25 &    16 &  78.60 & 19.64 &  43.98 &  93.37 \\
    & 0.30 &    16 & 118.54 & 22.28 &  62.10 & 132.78 \\
    & 0.35 &    16 &   1.13 &  1.35 &   0.06 &   5.15 \\
    & 0.40 &    16 &   0.23 &  0.44 &   0.01 &   1.27 \\
    & 0.45 &    16 &   0.12 &  0.45 &   0.01 &   1.79 \\
    & 0.50 &    16 &   0.07 &  0.17 &   0.01 &   0.51 \\
    \hline
    11 & 0.25 &    16 &  88.02 &  4.10 &  81.77 &  97.74 \\
    & 0.30 &    16 & 114.09 &  4.98 & 107.94 & 125.77 \\
    & 0.35 &    16 & 162.70 &  5.19 & 152.71 & 170.28 \\
    & 0.40 &    16 &   0.91 &  1.08 &   0.15 &   4.65 \\
    & 0.45 &    16 &   0.09 &  0.05 &   0.03 &   0.24 \\
    & 0.50 &    16 &   0.04 &  0.02 &   0.02 &   0.08 \\
    \hline
    14 & 0.25 &    16 &  84.17 &  2.55 &  79.04 &  89.22 \\
    & 0.30 &    16 & 114.12 &  3.90 & 106.48 & 122.46 \\
    & 0.35 &    16 & 161.89 &  3.86 & 155.45 & 168.40 \\
    & 0.40 &    16 & 240.00 &  0.00 & 240.00 & 240.00 \\
    & 0.45 &    16 &   1.13 &  0.91 &   0.09 &   2.74 \\
    & 0.50 &    16 &   0.47 &  0.61 &   0.03 &   1.66 \\
    \hline
    17 & 0.25 &    16 &  84.36 &  2.48 &  81.30 &  90.54 \\
    & 0.30 &    16 & 118.22 &  4.54 & 112.19 & 125.47 \\
    & 0.35 &    16 & 171.89 & 18.89 & 156.47 & 240.00 \\
    & 0.40 &    16 & 242.97 &  4.18 & 240.00 & 251.32 \\
    & 0.45 &    16 &  35.76 & 29.44 &   1.51 &  86.20 \\
    & 0.50 &    16 &   2.28 &  2.39 &   0.24 &   8.69 \\ 
    \hline
    20 & 0.25 &    16 &  86.28 &  3.13 &  82.21 &  92.09 \\
    & 0.30 &    16 & 116.27 &  2.99 & 111.47 & 122.35 \\
    & 0.35 &    16 & 164.82 &  2.70 & 159.52 & 169.72 \\
    & 0.40 &    16 & 241.38 &  3.23 & 240.00 & 250.92 \\
    & 0.45 &    16 & 240.00 &  0.00 & 240.00 & 240.00 \\
    & 0.50 &    16 & 240.00 &  0.00 & 240.00 & 240.00 \\
    \bottomrule
  \end{tabular}
  \caption{Backtacking Search with Shuffling and Least Used Variable
    heuristic}
\end{table}
\par
I chose the conditions to
\section{Results}
\section{Conclusions}

\end{document}